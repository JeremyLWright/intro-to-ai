%\documentclass[journal,10pt,onecolumn,draftcls]{IEEEtran}
\documentclass[12pt,letterpaper]{article}
\usepackage{listings}
\usepackage{wrapfig}
\usepackage{graphicx}
\usepackage{url}
\usepackage{setspace}
\usepackage[hidelinks]{hyperref}
\usepackage[style=ieee,backend=biber]{biblatex}
\usepackage[protrusion=true,expansion=true]{microtype}

\usepackage{float}
\usepackage{fullpage}
\addbibresource{WrightJeremy.bib}
\title{Artificial Intelligence's impact on Medical Diagnosis}
\author{Jeremy Wright}
\onehalfspacing
\begin{document}

% make the title area
\maketitle


Dr. Andrew Weil of the Arizona Center for Integrative Medicine states, ``If we
can make the correct diagnosis, the healing can begin. If we can't, both our
personal health and our economy are doomed. \autocite{Weil2009}'' Accurate,
traceable, and informed diagnosis are paramount to the health of patients.
Medicine, within the United States, is an enormous \$2.5 Trillion
opportunity and with the proliferation of research over many decades new medical
findingss are being published than could ever be grokked,
unassisted, by a doctor \autocite{Hays2012}.  Current advances in data
mining have moved to alleviate the shear volume of information available to
doctors. Medical data mining offers doctors the tools to distinguish anomalies from
trending epidemics; the tools to make evidence based decisions, ultimately leading to more personalized
medicine \autocite{Hays2012}. Hays, describes a tower of Data Mining
\autocite{Hays2012}. The base
of this tower is the raw data within the medical field, the images, structured
records and unstructured doctors notes. This information is processed, and
filtered into patterns and clusters. These clusters form the Information tier of
Hays' tower. Information here is a more refined form of the data, yet remain fully
traceable back to the original source. This Information is then processed again
to extract knowledge. Knowledge then leads to the top tier of Hays' model:
Decisions. Decisions are the actions of a diagnosis. Once a doctor
arrives to a diagnosis, as Dr. Weil states, ``healing can begin\ldots''. How can
this theoritical model be realized? And once prototypes are available, how can such a system
be rolled out to doctors without losing the valuable Knowledge tier?
One such system, offered by IBM, is a predictive analysis engine which offers
probable diagnosis with cited reasoning \autocite{Guazzelli2011}, and its
internal representation language Predictive Model Markup Language
\autocite{WikiPredictiveLanguage}.

Predictive Analysis is a branch of data-mining to make predictions about some
future event \autocite{WikiPredictiveAnalysis}. Predictive Analytics are
typified by their use of a scorecard. One of the more common
experiences with such a scorecard, at least in the United States, is that of
the FICO credit score, which is a single number or \emph{score} describing
a person's likelihood to default credit \autocite{Guazzelli2011}. A scorecar is
predictive analysis' greatest strength as a model. While more powerful, non-linear, models such as
neural-networks are capable of spectacular predictive results, neural networks do
not provide a traceable artifact to substantiate their predictions
\autocite{Segaran2008}. In the face
of malpractice and other fault finding behaviors, a traceable score card is
invaluable. With such powerful tools, how do we assemble the initial Data tier of
Hays' tower: Digital Records.

The 
medical industry has been slow to incorporate digital record keeping, yet
current legislation incentivized  the practice. Now, digital records emanate from
every facet of medicine. The uptake of these digital records is quickly enabling
\emph{Big Data} technologies of all branches to exercise their might. Yet, the
biggest impact these models can make is that in rural areas where doctors might
not see a volume of patiences sufficient to judge trends or induce the proper
research for truly unique cases. Predictive models, plugged into the global
economy of data can offer even small town doctors exceedingly more accurate and
better cited diagnosis \autocite{Guazzelli2011}. Despite the potential power of
these tools, the availability of systems for production use is meager. Hayes,
cites that the translation from ``scientists' desk'' to the doctors' hand is
fraught with technical challenges ranging from privacy concerns, to IT roll out.
During these challenges the Knowledge of these systems is sometimes lost. 

IBM, offers such a solution to preserve Knowledge by providing a standard
schema for the predictive models: \emph{Predictive Model Markup Language}
(PMML). This language derived from XML, is a de facto standard method for
encoding the Knowledge leveraged by these predictive systems
\autocite{WikiPredictiveLanguage}. Independent of
the IT processes, database vendor, or operating systems providing doctors access
to these platform PMML preserves Knowledge during both storage and exchange. 

Data Mining, has already made a profound impact on medicine by allowing doctors
to make informed, cited decisions through highly processed and filtered information,
yet the best of the technology is still yet to come. Digital records
are still coming into being, and even with the global economy, medicine is still
very closed within state governments. Beyond the international privacy and standardization
hurdles lie a nirvana of global medical knowledge, Knowledge allowing doctors to improve
the health of humanity. 



\printbibliography

\end{document}
