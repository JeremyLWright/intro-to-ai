% --------------------------------------------------------------
% This is all preamble stuff that you don't have to worry about.
% Head down to where it says "Start here"
% --------------------------------------------------------------
 
\documentclass[12pt]{article}
 
\usepackage[margin=1in]{geometry} 
\usepackage{amsmath,amsthm,amssymb,algpseudocode,listings}
\usepackage{color}
\usepackage{DejaVuSansMono} 
\usepackage{setspace}
\usepackage{parskip}
\usepackage{graphicx}


\definecolor{Code}{rgb}{0,0,0}
\definecolor{Decorators}{rgb}{0.5,0.5,0.5}
\definecolor{Numbers}{rgb}{0.5,0,0}
\definecolor{MatchingBrackets}{rgb}{0.25,0.5,0.5}
\definecolor{Keywords}{rgb}{0,0,1}
\definecolor{self}{rgb}{0,0,0}
\definecolor{Strings}{rgb}{0,0.63,0}
\definecolor{Comments}{rgb}{0,0.63,1}
\definecolor{Backquotes}{rgb}{0,0,0}
\definecolor{Classname}{rgb}{0,0,0}
\definecolor{FunctionName}{rgb}{0,0,0}
\definecolor{Operators}{rgb}{0,0,0}
\definecolor{Background}{rgb}{0.98,0.98,0.98}

\lstnewenvironment{python}[1][]{
\lstset{
numbers=left,
numberstyle=\ttfamily,
numbersep=1em,
xleftmargin=1em,
framextopmargin=2em,
framexbottommargin=2em,
showspaces=false,
showtabs=false,
showstringspaces=false,
frame=l,
tabsize=4,
% Basic
basicstyle=\ttfamily\small\setstretch{1},
backgroundcolor=\color{Background},
language=Python,
% Comments
commentstyle=\color{Comments}\slshape,
% Strings
stringstyle=\color{Strings},
morecomment=[s][\color{Strings}]{"""}{"""},
morecomment=[s][\color{Strings}]{'''}{'''},
% keywords
morekeywords={import,from,class,def,for,while,if,is,in,elif,else,not,and,or,print,break,continue,return,True,False,None,access,as,,del,except,exec,finally,global,import,lambda,pass,print,raise,try,assert},
keywordstyle={\color{Keywords}\bfseries},
% additional keywords
morekeywords={[2]@invariant},
keywordstyle={[2]\color{Decorators}\slshape},
emph={self},
emphstyle={\color{self}\slshape},
%
}}{}

\newcommand{\N}{\mathbb{N}}
\newcommand{\Z}{\mathbb{Z}}
 
\newenvironment{theorem}[2][Theorem]{\begin{trivlist}
\item[\hskip \labelsep {\bfseries #1}\hskip \labelsep {\bfseries #2.}]}{\end{trivlist}}
\newenvironment{lemma}[2][Lemma]{\begin{trivlist}
\item[\hskip \labelsep {\bfseries #1}\hskip \labelsep {\bfseries #2.}]}{\end{trivlist}}
\newenvironment{exercise}[2][Exercise]{\begin{trivlist}
\item[\hskip \labelsep {\bfseries #1}\hskip \labelsep {\bfseries #2.}]}{\end{trivlist}}
\newenvironment{problem}[2][Problem]{\begin{trivlist}
\item[\hskip \labelsep {\bfseries #1}\hskip \labelsep {\bfseries #2.}]}{\end{trivlist}}
\newenvironment{question}[2][Question]{\begin{trivlist}
\item[\hskip \labelsep {\bfseries #1}\hskip \labelsep {\bfseries #2.}]}{\end{trivlist}}
\newenvironment{corollary}[2][Corollary]{\begin{trivlist}
\item[\hskip \labelsep {\bfseries #1}\hskip \labelsep {\bfseries #2.}]}{\end{trivlist}}


\lstset{basicstyle=\footnotesize\ttfamily,breaklines=true}
\lstset{framextopmargin=50pt,frame=bottomline}

\begin{document}
 
% --------------------------------------------------------------
%                         Start here
% --------------------------------------------------------------
 
\title{Homework 1}%replace X with the appropriate number
\author{Jeremy Wright\\ %replace with your name
CSE571 - Artificial Intelligence} %if necessary, replace with your course title
 
\maketitle
\begin{problem}{1}
 
Using the Initial values provided, calculate all join probabilities.

\begin{tabular}{|l|l|l|l|}
        \hline
        Joint Probability & Expansion  & Initial Result & X Counts\\
        \hline
        \hline
        P(Cherry, Red, H, B=1)    & $ \theta \theta_{W1} \theta_{H1} \theta_{F1}$                             & 0.0864 & 189\\ \hline
        P(Cherry, Red, H, B=2)    & $ (1-\theta) \theta_{W2} \theta_{H2}       \theta_{F2}$                   & 0.384  & 84\\ \hline
        P(Cherry, Red, !H, B=1)   & $ \theta \theta_{W1} (1-\theta_{H1})       \theta_{F1}$                   & 0.1296 & 64.38\\ \hline
        P(Cherry, Red, !H, B=2)   & $ (1-\theta) \theta_{W1} (1-\theta_{H2})       \theta_{F2}$               & 0.0576 & 28.61\\ \hline
        P(Cherry, Green, H, B=1)  & $ \theta (1-\theta_{W1}) \theta_{H1}       \theta_{F1}$                   & 0.0576 & 52\\ \hline
        P(Cherry, Green, H, B=2)  & $ (1-\theta) (1-\theta_{W2}) \theta_{H2}       \theta_{F2}$               & 0.0576 & 52\\ \hline
        P(Cherry, Green, !H, B=1) & $ \theta (1-\theta_{W1}) (1-\theta_{H1})       \theta_{F1}$               & 0.0864 & 45\\ \hline
        P(Cherry, Green, !H, B=2) & $ (1-\theta) (1-\theta_{W2}) (1-\theta_{H2})       \theta_{F2}$           & 0.0864 & 45\\ \hline
        P(Lime, Red, H, B=1)      & $ \theta \theta_{W1} \theta_{H1}      (1-\theta_{F1})$                    & 0.0576 & 54.69\\ \hline
        P(Lime, Red, H, B=2)      & $ (1-\theta) \theta_{W2} \theta_{H2}(1-\theta_{F2})$                      & 0.0256 & 24.30\\ \hline
        P(Lime, Red, !H, B=1)     & $ \theta \theta_{W1} (1-\theta_{H1})     (1-\theta_{F1})$                 & 0.0864 & 69.23\\ \hline
        P(Lime, Red, !H, B=2)     & $ (1-\theta) \theta_{W2} (1-\theta_{H2}) (1-\theta_{F2})$                 & 0.0384 & 30.76\\ \hline
        P(Lime, Green, H, B=1)    & $ \theta (1-\theta_{W1}) \theta_{H1}    (1-\theta_{F1})$                  & 0.0384 & 47\\ \hline
        P(Lime, Green, H, B=2)    & $ (1-\theta) (1-\theta_{W2}) \theta_{H2} (1-\theta_{F2})$                 & 0.0384 & 47\\ \hline
        P(Lime, Green, !H, B=1)   & $ \theta (1-\theta_{W1}) (1-\theta_{H1})(1-\theta_{F1})$                  & 0.0576 & 83.5\\ \hline
        P(Lime, Green, !H, B=2)   & $ (1-\theta) (1-\theta_{W2}) (1-\theta_{H2})(1-\theta_{F2})$              & 0.0576 & 83.5\\ \hline
    \end{tabular}

    We can then proceed with the E step, to calculate all the new theta values.

\begin{tabular}{|l|c|l|}
        \hline
        Next $\theta$ Value & X Expansion & Initial Result\\
        \hline
        \hline
        $ \theta^1 $      & $\dfrac{x_1 + x_3 + x_5 + x_7 + x_{11} + x_{13} + x_{15}}{x_1+x_2+\dots+x_{16}}$ & 0.604\\ \hline
        $ \theta_{F1}^1 $ & $\dfrac{x_1+x_3 + x_5+x_7 }{x_1+x_2+\dots+x_{8} }$                               & 0.625\\ \hline
        $ \theta_{F2}^1 $ & $\dfrac{x_2+x_4+x_8}{ x_1+x_2+\dots+x_{8}}$                                      & 0.374\\ \hline
        $ \theta_{W1}^1 $ & $\dfrac{x_1+x_3+x_9+x_{11} }{x_1 +\dots+x_4 + x_9+\dots+x_{12} }$                & 0.692\\ \hline
        $ \theta_{W2}^1 $ & $\dfrac{x_2+x_4+x_{10}+x_{12} }{x_1 +\dots+x_4 + x_9+\dots+x_{12} }$             & 0.307\\ \hline
        $ \theta_{H1}^1 $ & $\dfrac{x_1+x_5+x_9+x_{13} }{ x_1+x_2+x_5+x_6+x_9+x_{10}+x_{13}+x_{14}}$         & 0.623\\ \hline
        $ \theta_{H2}^1 $ & $\dfrac{x_2+x_6+x_{10} + x_{14}}{x_1+x_2+x_5+x_6+x_9+x_{10}+x_{13}+x_{14} }$     & 0.376\\ \hline
    \end{tabular}

    With the new theta values, we can continue in this fashion until we minimize all $\theta^0 - \theta_1$. Excel's Genetic Solver can do this effectively
    and results in the following table.

    \begin{tabular}{|l|c|}
        \hline
        Next $\theta$ Value & Iterated Result\\
        \hline
        \hline
        $ \theta^1 $      &  0.504\\ \hline
        $ \theta_{F1}^1 $ &  0.416\\ \hline
        $ \theta_{F2}^1 $ &  0.264\\ \hline
        $ \theta_{W1}^1 $ &  0.281\\ \hline
        $ \theta_{W2}^1 $ &  0.744\\ \hline
        $ \theta_{H1}^1 $ &  0.561\\ \hline
        $ \theta_{H2}^1 $ &  0.410\\ \hline
    \end{tabular}

    Lastly, using the iterated theta values, we result in the following counts of candy. This represents the most likely split of candies from each bag.

\begin{tabular}{|l|l|l|}
        \hline
        Joint Probability & Final Result & X Counts\\
        \hline
        \hline
        P(Cherry, Red, H, B=1)    &  0.033 & 123.54\\ \hline
        P(Cherry, Red, H, B=2)    &  0.040 & 149.45\\ \hline
        P(Cherry, Red, !H, B=1)   &  0.025 & 28.85\\ \hline
        P(Cherry, Red, !H, B=2)   &  0.057 & 64.12\\ \hline
        P(Cherry, Green, H, B=1)  &  0.084 & 89.42\\ \hline
        P(Cherry, Green, H, B=2)  &  0.013 & 14.57\\ \hline
        P(Cherry, Green, !H, B=1) &  0.066 & 69.25\\ \hline
        P(Cherry, Green, !H, B=2) &  0.019 & 20.74\\ \hline
        P(Lime, Red, H, B=1)      &  0.046 & 23.27\\ \hline
        P(Lime, Red, H, B=2)      &  0.111 & 55.72\\ \hline
        P(Lime, Red, !H, B=1)     &  0.036 & 18.51\\ \hline
        P(Lime, Red, !H, B=2)     &  0.159 & 81.48\\ \hline
        P(Lime, Green, H, B=1)    &  0.118 & 71.07\\ \hline
        P(Lime, Green, H, B=2)    &  0.382 & 22.92\\ \hline
        P(Lime, Green, !H, B=1)   &  0.092 & 104.84\\ \hline
        P(Lime, Green, !H, B=2)   &  0.055 & 62.15\\ \hline
    \end{tabular}



\end{problem}

\end{document}
